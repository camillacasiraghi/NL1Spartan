\documentclass[12pt,oneside,a4paper]{article}

\usepackage[backend=biber,style=numeric]{biblatex}
\usepackage{xcolor}
\usepackage{todonotes}
\usepackage{amsmath}
\usepackage{multicol}
\usepackage{caption}
\usepackage{hyperref}
\usepackage{graphicx}
\usepackage{listings}
\lstset{
	frame=top,frame=bottom,
	language=C,
	basicstyle=\small\normalfont,
	xleftmargin=\parindent,
	keywordstyle=\color{green!40!black},
	%  commentstyle=\itshape\color{purple!40!black},
	%  identifierstyle=\color{blue},
	%  stringstyle=\color{orange},
	morekeywords={in, globaldata, procedure, input, output, behavior, end, XOR, NOT, AND}, % keyword to highlight
	%  captionpos=t,
	tabsize=2,
	numbers=left,
	stepnumber=1,                   % the step between two line-numbers.        
	numbersep=5pt,
	framexleftmargin=10pt,
	title=\lstname,
	captionpos=t,
	showspaces=false,
}
\DeclareCaptionFormat{listing}{\rule{\dimexpr\textwidth\relax}{0.4pt}\par\vskip1pt#1#2#3}
\captionsetup[lstlisting]{format=listing,singlelinecheck=false, margin=0pt,labelsep=space,labelfont=bf}

\usepackage{booktabs}
\usepackage[noabbrev,capitalise]{cleveref}
\crefname{listing}{algorithm}{algorithms}
\Crefname{listing}{Algorithm}{Algorithms}
\renewcommand\lstlistingname{Algorithm}
\def\lstlistingcrefname{Algorithm}
\usepackage{url}

\addbibresource{biblio.bib}

\title{\textbf{Spartan and Learning NL1 \\ Plan, Strategies and Goals}}

\author{Camilla Casiraghi}

\date{2023}

\begin{document}


\begin{titlepage}
	\centering
	\clearpage
	\maketitle
	\thispagestyle{empty}
	\vspace*{1cm}
	\vfill
	\centering
	\includegraphics{logo_polimi.png}\includegraphics{logo_NECST.png}
\end{titlepage}


\begin{abstract}
This paper explores the parallels between the experience of Spartan races and the process of learning. Within the context of Spartan Races, sporting events that involve obstacle course races of varying distances and difficulties, several analogies can be drawn to the academic learning journey. This article examines the similarities between the different types of Spartan races and various levels of learning, highlighting how both require commitment, personal challenges, and strategic preparation. It delves into the concept of deliberate practice and performance, emphasizing the importance of setting appropriate objectives and having the right mindset in both Spartan races and academic pursuits. Furthermore, it discusses the role of self-reflection, self-compassion, and celebrating milestones in fostering growth and accomplishment in both domains. By drawing parallels between Spartan races and learning, this paper offers insights into the mindset and strategies that can enhance the learning experience and inspire individuals to overcome challenges and achieve their goals.
\end{abstract}

\section{Introduction} \label{sec:intro}
Physical activity holds a significant place in the NecstCAMP pyramid, just below the realm of learning. Activities such as Spartan races can be compared to academic courses at the Politecnico. But what connects the world of Spartan races to the academic environment? In this paper, we will explore the parallelism between Spartan races and learning, highlighting their common elements and how they contribute to personal growth and achievement.
\begin{figure}[h]
    \centering
    \includegraphics[width=.7\textwidth]{piramide.png}
    \caption{The pyramid of the NecstCAMP}
    \label{fig:my_label}
\end{figure}

\section{Spartan Races} \label{sec:spraces}
Spartan races come in three different categories: Sprint (5km with 20 obstacles), Super (10km with 25 obstacles), and Beast (21km with 30 obstacles). Skipping an obstacle results in penalties such as burpees or penalty loops. Additionally, there are other race variations, including timed races, stadium races, and city races. Interestingly, the number of obstacles does not proportionally increase with the distance covered. These obstacles range from carrying heavy objects and crawling under barbed wire to climbing ropes and scaling walls.

\section{The Spartan Experience} \label{sec:spexp}
Spartan races excel at engaging participants through various means. At the end of each race, participants are rewarded with a t-shirt and a medal (red for Sprint, blue for Super, and green for Beast). Moreover, completing all three race categories within a year earns a Trifecta, symbolized by a one-third circular wedge that, when combined, forms a complete medal. Achieving the Trifecta within a calendar year grants eligibility for the Trifecta Weekend in Sparta. Multiple Trifectas lead to progressively larger medals, representing the number of Trifectas completed. Sparta hosts the Trifecta Weekend in November, which serves as the World Championship. To participate in the Beast category during this event, one must also complete the Super and Sprint races. It is a moment of celebration, honoring athletes who have closed the highest number of Trifectas in their history, such as 30 or 50. Furthermore, if a participant completes 13 Trifectas in a year, they are awarded a shield during the Shield ceremony at Sparta. Each year, the medals feature different symbols, such as a scorpion or jaguar, representing the Spartan theme.

\section{NecstCAMP and Spartan Race Partnership} \label{sec:partn}
NecstCAMP collaborates with Spartan Race Italy, offering students the opportunity to participate in various events throughout the year. Not all events are Trifecta weekends, and there is also a non-competitive category for kids. This partnership allows NecstCAMP students to be involved both as volunteers and as runners. They can alternate between participating in races and volunteering on different weekends. This special relationship between NecstCAMP and Spartan Race stands out, as normally, volunteers usually focus solely on their volunteering duties for a weekend. 

Every volunteer, including NecstCAMP students, receives vouchers for future races after each weekend. However, NecstCAMP students are not allowed to reuse those vouchers, fostering a mutual trust between NecstCAMP and the Spartan Race community.

\section{Training and Progression} \label{sec:prog}
As students engage in successive Spartan races, they are gradually qualified for more challenging courses through a deliberate educational process. They start by running the Sprint race on the first weekend, then have the opportunity to run both the Sprint and Super races on the following weekend. Finally, they become qualified for the Beast race. The idea is that if it is the first year participating in Spartan races, the Trifecta cannot be completed within that year, creating a desire to achieve the milestone without rushing. The game becomes a challenge with oneself.

\section{Categories and Teamwork} \label{sec:cat}
Spartan races have three categories: elite, age, and open. Elite participants are professionals, while the age category includes professionals divided by age groups. NecstCAMP students participate in the open category, which is non-competitive unlike the others. During the race, the challenge with oneself combines with teamwork. Playing this challenge with oneself involves making everything measurable, observable, and repeatable. Each race sets a new goal, considering the different conditions presented in various races.

\section{Preparing for Challenges} \label{sec:prep}
To arrive prepared for a challenge, one must train through crossfit courses, saturday training with Coach Andal, and SGX courses at the Politecnico. Returning to the parallelism between Spartan races and courses, we can identify a key similarity: the phases of practice and performance. Through practice, we apply what we learn and adjust our practice based on what happens during performance. This iterative process of deliberate practice enhances our capabilities. The more we practice, the more skilled we become. 

This principle applies not only to sports but also to academic exams. Deliberate practice involves stepping out of our comfort zone and challenging ourselves with tasks slightly beyond our current abilities, allowing us to gradually expand our potential. In contrast, during performance, we push ourselves beyond our comfort zone to tap into our full potential. However, it is crucial to properly challenge ourselves; the game only works when the challenge matches our abilities, neither too easy nor too difficult.

\section{The Importance of Performance} \label{sec:imp}
During a performance, both the body and mind progress as they precisely execute the required tasks. In contrast, during practice, we maintain full awareness and consciousness. 

The practice in Spartan races corresponds to CrossFit workouts, while in the academic context, it aligns with courses and exercise hours. The performance phase for Spartan races corresponds to the actual races, while in the academic realm, it aligns with exams. Consider what would happen if one attempted to lift a 50kg ball for the first time during a Spartan race or take an exam after a sleepless night. It is natural to feel nervous about performance situations such as exams or projects.

\section{Setting Expectations and Defining Goals} \label{sec:sett}
One significant aspect of performance is the act of applying what we have learned. Performance provides an opportunity to expand our potential by facing appropriate challenges. But what constitutes a properly challenge? It is something that allows us to set clear objectives, as demonstrated at NecstCAMP when defining personal objectives. To arrive prepared for a challenge, one must train properly and select the properly challenge based on their current abilities, setting the right expectations. The right expectation translates into defining 3 objectives: minimum, realistic, and wow. The minimum represents the goals that we absolutely do not want to fall below, considering all the external conditions that may go wrong and are beyond our control; the realistic goal is what I want to aim for; the wow goal is what excites us because we have achieved it. 

This parallels the world of Spartan races, where running a race and training are analogous to taking exams and preparing for them. Defining goals is crucial to create the properly challenge, both in Spartan races (measured by time and saved obstacles) and in exams (measured by credits and grades). It is equally important to develop a plan to tackle the challenge strategically, ensuring we do not exhaust all our resources immediately. During a performance, there are sensations that can compromise future performances; for instance, running the first race but having two more races the next day will lead to fatigue. Similarly, refusing to take an exam now, but having another on the same day, will result in additional difficulties. Making impulsive decisions during the performance is not ideal; therefore, creating and adhering to a well-thought-out plan is of maximum importance.
\begin{figure}[h]
    \centering
    \includegraphics[width=.7\textwidth]{tabella.png}
    \caption{Example of objectives}
    \label{fig:my_label}
\end{figure}

\section{Conclusion} \label{sec:concl}
By reflecting on our performance, we can assess if we have assigned ourselves the properly challenge and goals. We must learn to appreciate and value ourselves, as we are often our own harshest critics. Remembering where we started helps us enjoy the journey and be proud of our achievements. In the parallel worlds of Spartan races and learning, practice, and performance, setting expectations, defining goals, and strategizing are all essential elements that contribute to personal growth, success, and self-appreciation.
\newpage
\title{\textbf{Bibliography}} \\
\begin{itemize}
\item NecstCAMP meeting of 14/11/2022 held by professor Marco D. Santambrogio
\item NecstCAMP meeting of 25/03/2023 held by professor Marco D. Santambrogio
\end{itemize}



\end{document}


